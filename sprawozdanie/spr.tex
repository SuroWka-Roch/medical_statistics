\documentclass[a4paper,11pt]{article}
\usepackage[utf8]{inputenc}
%\usepackage[T1]{polski}
\usepackage{helvet}
\usepackage{graphicx}
\usepackage{float}

\usepackage{color}
\usepackage{geometry}
\usepackage[T1]{fontenc}
\usepackage[unicode]{hyperref}
\usepackage{amsmath}
\usepackage{gensymb}
\usepackage{multirow}
\geometry{hmargin={2cm, 2cm}, height=10.0in}


\usepackage{lipsum} 
\usepackage{indentfirst}

\usepackage{listings}
\usepackage{color}

\definecolor{dkgreen}{rgb}{0,0.6,0}
\definecolor{gray}{rgb}{0.5,0.5,0.5}
\definecolor{mauve}{rgb}{0.58,0,0.82}

\lstset{frame=tb,
  language=Java,
  aboveskip=3mm,
  belowskip=3mm,
  showstringspaces=false,
  columns=flexible,
  basicstyle={\small\ttfamily},
  numbers=none,
  numberstyle=\tiny\color{gray},
  keywordstyle=\color{blue},
  commentstyle=\color{dkgreen},
  stringstyle=\color{mauve},
  breaklines=true,
  breakatwhitespace=true,
  tabsize=3
}

\title{Title}
\author{Wojciech Surówka}

\begin{document}

\begin{table}[t]
\centering
\label{my-label}
\resizebox{\textwidth}{!}{
\begin{tabular}{|c|c|l|l|l|l|}
\hline

\begin{tabular}[c]{@{}c@{}}Wydział: FiIS\\ Kierunek: FM\end{tabular} & \begin{tabular}[c]{@{}c@{}}Rok: II\\ Zespół: 2\end{tabular} & \multicolumn{4}{l|}{\begin{tabular}[c]{@{}l@{}}Zofia Pieńkowska \\ Wojciech Surówka\end{tabular}} \\ \hline

\textbf{\begin{tabular}[c]{@{}c@{}} \end{tabular}} & \multicolumn{5}{l|}{\begin{tabular}[c]{@{}l@{}}Temat:Statystyka w Medycynie: projekt cz. 1\end{tabular}} \\ \hline

\begin{tabular}[c]{@{}c@{}}Data wykonania:\\ 1.11.2019 r.\end{tabular} & \begin{tabular}[c]{@{}c@{}}Data oddania:\\ 14.11.2019 r.\end{tabular} & \begin{tabular}[c]{@{}l@{}}Zwrot do poprawy:\\\end{tabular} & \begin{tabular}[ct]{@{}l@{}}Data oddania poprawy:\\\end{tabular} & \begin{tabular}[cT]{@{}l@{}}Data zaliczenia:\\\end{tabular} & \begin{tabular}[cT]{@{}l@{}}Ocena:\end{tabular} \\ \hline

\end{tabular}}
\end{table}

\section{Temat wspólny}

Dane główne:
\begin{table}[H]
\begin{center}
\caption{Główne informacje o bazie danych}
\begin{tabular}{cc}
\hline
Liczba pacjentów w bazie: & 6861 \\
liczba badań w bazie: & 17198 \\
liczba odnalezionych rodzajów & 15553 \\
Liczba wykonanych badań oporności & 136286  \\
\hline
\end{tabular}
\end{center}
\end{table}
Wszelkie analizy wykonane zostały przy zastosowaniu języka programowania Python oraz biblioteki Pandas.

Wykonano wstępną obróbkę danych:
\begin{itemize}
  \item Uwzględniono obecność pacjenta na wielu oddziałach (7),
  \item usunięto powtórzenia w zapisie pacjentów (2), 
  \item usunięto badania o błędnym zapisie daty,
  \item uwzględniono badania w przedziale 1-1-1992 do 31-3-1996 włącznie,
  \item usunięto dane drobnoustroju niezwiązanego z żadnym badaniem,
  \item usunięto dane oporność niezwiązane z żadnym drobnoustrojem.
\end{itemize}

Po przeprowadzeniu tych działań otrzymano zmienioną wielkość tablic:
\begin{table}[H]
\begin{center}
\caption{Zmiany wielkości danych po wstępnej analizie}
\begin{tabular}{c|c|c}
\hline
Nazwa danych & Nowa ilość & zmiana \\
Pacjenci & 6852 & 9 \\
Badania & 16995  & 203\\
Drobnoustroju & 15421 & 132\\
Lekooporność & 135220 & 1066 \\
\hline
\end{tabular}
\end{center}
\end{table}

Na podstawie materiałów (wymaz z pochwy lub sperma) ustalono, że w danych pacjenta liczba 1 oznacza mężczyznę, a 0 kobietę.

Ustalono, że badano jedynie 41 rodzajów materiałów. Wykryto 178 różnych drobnoustrojów. Testowano 78 różnych leków. W 8 przypadkach nie podano nazwy oddziału, ponadto nazwy wydziałów są podawane zamiennie z małej i wielkiej litery.
W celu unifikacji wszystkie wydziały, których nazwę podano z małej litery, zapisano wielkimi literami.

Wyniki wstępnego szacowania umieszczono w tabelach poniżej.

\begin{table}[H]
\begin{center}
\caption{Wstępnie oszacowane dane.}
\begin{tabular}{|c|c|c|}
\hline
Typ pacjentów &  co najmniej jedno badanie & co najmniej jeden drobnoustrój \\
\hline
Wszyscy & 6746 &5189\\
kobiety  & 3165  & 2428\\
Mężczyźni & 3581 & 2761\\ \hline
\multicolumn{3}{|c|}{Rok 1992} \\ \hline
Wszyscy & 2150 &1599\\
kobiety  & 958  & 698\\
Mężczyźni & 1192 & 901\\ \hline
\multicolumn{3}{|c|}{Rok 1993} \\ \hline
Wszyscy & 1367 &1282\\
kobiety  & 690  & 630\\
Mężczyźni & 677 & 652\\ \hline
\multicolumn{3}{|c|}{Rok 1994} \\ \hline
Wszyscy & 1350 &1173\\
kobiety  & 646  & 563\\
Mężczyźni & 704 & 610\\ \hline
\multicolumn{3}{|c|}{Rok 1995} \\ \hline
Wszyscy & 1864 &1254\\
kobiety  & 888  & 613\\
Mężczyźni & 976 & 641\\ \hline
\multicolumn{3}{|c|}{Rok 1996} \\ \hline
Wszyscy & 559 &405\\
kobiety  & 258  & 191\\
Mężczyźni & 301 & 214\\ \hline
\hline
\end{tabular}
\end{center}
\end{table}

\begin{table}[H]
\begin{center}
\caption{Wykaz badań dla wszystkich oddziałów}
\begin{tabular}{c|c|c}
\hline
Rok badań & Ilość badań & Ilość badań \\
& z przynajmniej jednym drobnoustrojem &z testem lekooporności \\
1992&5106&2327 \\
1993&3097&1992 \\
1994&3186&2120 \\
1995&4346&2126 \\
1996&1260&704 \\
\hline
\end{tabular}
\end{center}
\end{table}

\begin{table}[H]
\begin{center}
\caption{Wykaz badań dla poszczególnych oddziałów}
\begin{tabular}{c|c|c}
\hline
Rok badań & Ilość badań & Ilość badań \\
& z przynajmniej jednym drobnoustrojem &z testem lekooporności \\
\hline \multicolumn{3}{|c|}{Oddział OIOM} \\ \hline
wspólnie &6043 &3546 \\
1992 &1649 &893 \\
1993 &1210 &785 \\
1994 &1297 &843 \\
1995 &1447 &778 \\
1996 &440 &247 \\
\hline \multicolumn{3}{|c|}{Oddział NACZ} \\ \hline
wspólnie &428 &304 \\
1992 &160 &99 \\
1993 &123 &97 \\
1994 &51 &50 \\
1995 &79 &49 \\
1996 &15 &9 \\
\hline \multicolumn{3}{|c|}{Oddział URAZ} \\ \hline
wspólnie &1171 &683 \\
1992 &308 &139 \\
1993 &191 &144 \\
1994 &231 &174 \\
1995 &356 &176 \\
1996 &85 &50 \\
\hline \multicolumn{3}{|c|}{Oddział WEWN} \\ \hline
wspólnie &1848 &623 \\
1992 &822 &232 \\
1993 &251 &108 \\
1994 &293 &118 \\
1995 &401 &127 \\
1996 &81 &38 \\
\hline \multicolumn{3}{|c|}{Oddział CHIR} \\ \hline
wspólnie &622 &422 \\
1992 &180 &107 \\
1993 &87 &74 \\
1994 &125 &100 \\
1995 &179 &100 \\
1996 &51 &41 \\
\hline \multicolumn{3}{|c|}{Oddział REUM} \\ \hline
wspólnie &799 &372 \\
1992 &204 &73 \\
1993 &160 &88 \\
1994 &171 &107 \\
1995 &200 &78 \\
1996 &64 &26 \\

\hline
\end{tabular}
\end{center}
\end{table}

\begin{table}[H]
\begin{center}
\caption{Wykaz badań dla poszczególnych oddziałów c.d.}
\begin{tabular}{c|c|c}
\hline
Rok badań & Ilość badań & Ilość badań \\
& z przynajmniej jednym drobnoustrojem &z testem lekooporności \\

\hline \multicolumn{3}{|c|}{Oddział NEFR} \\ \hline
wspólnie &2039 &789 \\
1992 &511 &125 \\
1993 &343 &151 \\
1994 &312 &193 \\
1995 &693 &241 \\
1996 &180 &79 \\
\hline \multicolumn{3}{|c|}{Oddział UROL} \\ \hline
wspólnie &552 &320 \\
1992 &178 &83 \\
1993 &123 &87 \\
1994 &85 &59 \\
1995 &130 &65 \\
1996 &36 &26 \\
\hline \multicolumn{3}{|c|}{Oddział NECH} \\ \hline
wspólnie &649 &390 \\
1992 &195 &98 \\
1993 &117 &97 \\
1994 &122 &89 \\
1995 &148 &75 \\
1996 &67 &31 \\
\hline \multicolumn{3}{|c|}{Oddział NEUR} \\ \hline
wspólnie &681 &392 \\
1992 &290 &133 \\
1993 &161 &119 \\
1994 &85 &68 \\
1995 &105 &53 \\
1996 &40 &19 \\


\end{tabular}
\end{center}
\end{table}

\begin{table}[H]
\begin{center}
\caption{Wykaz badań dla poszczególnych oddziałów c.d.}
\begin{tabular}{c|c|c}
\hline
Rok badań & Ilość badań & Ilość badań \\
& z przynajmniej jednym drobnoustrojem &z testem lekooporności \\

\hline \multicolumn{3}{|c|}{Oddział USPR} \\ \hline
wspólnie &846 &555 \\
1992 &180 &103 \\
1993 &95 &70 \\
1994 &200 &139 \\
1995 &264 &174 \\
1996 &107 &69 \\
\hline \multicolumn{3}{|c|}{Oddział PLAS} \\ \hline
wspólnie &381 &286 \\
1992 &141 &95 \\
1993 &81 &66 \\
1994 &57 &51 \\
1995 &83 &59 \\
1996 &19 &15 \\
\hline \multicolumn{3}{|c|}{Oddział OKUL} \\ \hline
wspólnie &204 &101 \\
1992 &50 &17 \\
1993 &23 &10 \\
1994 &20 &14 \\
1995 &81 &40 \\
1996 &30 &20 \\
\hline \multicolumn{3}{|c|}{Oddział LARY} \\ \hline
wspólnie &605 &407 \\
1992 &214 &122 \\
1993 &110 &80 \\
1994 &103 &93 \\
1995 &140 &84 \\
1996 &38 &28 \\
\hline \multicolumn{3}{|c|}{Oddział SZCZ} \\ \hline
wspólnie &153 &95 \\
1992 &36 &17 \\
1993 &31 &20 \\
1994 &38 &25 \\
1995 &41 &27 \\
1996 &7 &6 \\

\end{tabular}
\end{center}
\end{table}

\newpage
\section{Temat indywidualny}

\begin{table}[H]
  \begin{center}
  \caption{Wykaz badań dla materiału mocz z podziałem na płeć}
  \begin{tabular}{|c|c|c|c|}
  \hline
  &Ilość materiału: mocz & Mocz + Drobnoustrój & Lekooporność \\ \hline
  Wspólnie &5945 &3698 &3496\\ \hline
Mężczyźni &2879 &1738 &1719\\ \hline
Kobiety &3066 &1960 &1777\\ \hline
  \end{tabular}
\end{center}
\end{table}

\begin{table}[H]
  \begin{center}
  \caption{Wykaz badań dla materiału mocz dla lat 1992-1996}
  \begin{tabular}{|c|c|c|c|}
  \hline
  Rok & Ilość materiału: mocz & Mocz + Drobnoustrój & Lekooporność \\ \hline 
  
\multicolumn{4}{|c|}{Wspólnie} \\ \hline
1992 &1718 &914 &805\\ \hline
1993 &1150 &926 &787\\ \hline
1994 &1053 &807 &801\\ \hline
1995 &1625 &832 &864\\ \hline
1996 &392 &215 &234\\ \hline



\multicolumn{4}{|c|}{Mężczyźni} \\ \hline
1992 &887 &473 &453\\ \hline
1993 &490 &410 &379\\ \hline
1994 &511 &388 &384\\ \hline
1995 &804 &380 &409\\ \hline
1996 &182 &85 &92\\ \hline



\multicolumn{4}{|c|}{Kobiety} \\ \hline
1992 &831 &441 &352\\ \hline
1993 &660 &516 &408\\ \hline
1994 &542 &419 &417\\ \hline
1995 &821 &452 &455\\ \hline
1996 &210 &130 &142\\ \hline
\end{tabular}
\end{center}
\end{table}


Określono liczbę badań na materiale mocz:
\begin{itemize}
\item OIOM:1321
\item NACZ:66
\item URAZ:428
\item WEWN:607
\item CHIR:69
\item REUM:487
\item NEFR:1694
\item UROL:333
\item NECH:145
\item NEUR:321
\item USPR:307
\item PLAS:53
\item OKUL:51
\item LARY:40
\item SZCZ:30
\end{itemize}
Na podstawie powyższej listy można wyciągnąć zgodny z intuicją wniosek, że najczęściej badano mocz na wydziale Nefrologii. Drugi w kolejności jest OIOM, co może wynikać z dostępności moczu w przypadku badań zatruć.  

\begin{table}[H]
  \begin{center}
  \caption{"Ilość badań dla materiału mocz dla lat 1992-1996 dla oddziału Nefrologii"}
  \begin{tabular}{|c|c|c|c|}
    \hline
    Rok & Wszyscy & Kobiety & Mężczyźni \\ \hline
1992 &416 &224 &192\\ \hline
1993 &295 &229 &66\\ \hline
1994 &263 &154 &109\\ \hline
1995 &567 &321 &246\\ \hline
1996 &153 &91 &62\\ \hline
  \end{tabular}
\end{center}
\end{table}


Odnaleziono najczęściej spotykane drobnoustroje w moczu. 4 najczęstsze w kolejności to:
\begin{itemize}
\item Escherichia coli 1 - 1115 wystąpień
\item Proteus mirabilis - 432 wystąpień
\item Enterococcus faecalis 1 - 332 wystąpień
\item Pseudomonas aeruginosa - 294 wystąpienia
\end{itemize}

Następnie przeprowadzono analizę dla dwóch pierwszych drobnoustrojów z podziałem na lata i płeć.

\begin{table}[H]
  \begin{center}
  \caption{Liczba badań dla materiału mocz: dla szczególnych drobnoustrojówz podziałem na lata}
  \begin{tabular}{|c|c|c|c|}
    \hline
    \multicolumn{4}{|c|}{Escherichia coli 1} \\ 
    \hline Rok &Wspólnie & Kobiety & Mężczyźni \\ \hline
    Każdy& 1110& 823& 287 \\ \hline 
    1992& 247& 169& 78 \\ \hline
    1993& 254& 197& 57 \\ \hline
    1994& 254& 196& 58 \\ \hline
    1995& 277& 199& 78 \\ \hline
    1996& 78& 62& 16 \\ \hline

    \multicolumn{4}{|c|}{Proteus mirabilis} \\ \hline
    Rok &Wspólnie & Kobiety & Mężczyźni \\ \hline
    Każdy& 432& 228& 204 \\ \hline 
    1992& 103& 60& 43 \\ \hline
    1993& 90& 50& 40 \\ \hline
    1994& 109& 60& 49 \\ \hline
    1995& 92& 34& 58 \\ \hline
    1996& 37& 24& 13 \\ \hline
  \end{tabular}
\end{center}
\end{table}

\newpage
Znaleziono najczęściej testowane leki – cztery najczęstsze to:
\begin{itemize}
  \item gentamycin : 1149
  \item ampicillin : 1120
  \item carbenicillin 2.5 : 837
  \item augmentin : 831
\end{itemize}

\begin{table}[h]
  \begin{center}
  \caption{Wykaz badań dla materiału mocz dla lat 1992-1996 względem używanych leków}
  \begin{tabular}{|c|c|c|c|}
    \hline
    \multicolumn{4}{|c|}{gentamycin} \\ 
    \hline Rok &Wspólnie & Kobiety & Mężczyźni \\ \hline
Każdy& 2075& 1107& 968 \\ \hline 
1992& 631& 292& 339 \\ \hline
1993& 516& 290& 226 \\ \hline
1994& 511& 292& 219 \\ \hline
1995& 281& 141& 140 \\ \hline
1996& 133& 91& 42 \\ \hline
\multicolumn{4}{|c|}{ampicillin} \\ 
\hline Rok &Wspólnie & Kobiety & Mężczyźni \\ \hline
Każdy& 744& 563& 181 \\ \hline 
1992& 138& 104& 34 \\ \hline
1993& 135& 101& 34 \\ \hline
1994& 192& 148& 44 \\ \hline
1995& 207& 153& 54 \\ \hline
1996& 72& 57& 15 \\ \hline
  \end{tabular}
\end{center}
\end{table}

Najczęściej testowana para dla materiału mocz i najpopularniejszego drobnoustroju to gentamycin z ampicillin – razem wystąpiły 911 razy;
 następne w kolejności jest augmentin z gentamycin; na trzecim miejscu gentamycin z carbenicillin 2.5.
Kolejne wnioski wyprowadzone są dla połączenia numerów id 12 i 27 (gentamycin z ampicillin).


\begin{table}[h]
  \begin{center}
  \caption{Ilość badań dla najpopularniejszego połączenia leków w wykazie lat}
  \begin{tabular}{|c|c|c|c|}
\hline Rok &Wspólnie & Kobiety & Mężczyźni \\ \hline
Każdy& 581& 434& 147 \\ \hline 
1992& 133& 99& 34 \\ \hline
1993& 119& 89& 30 \\ \hline
1994& 177& 134& 43 \\ \hline
1995& 80& 55& 25 \\ \hline
1996& 72& 57& 15 \\ \hline

  \end{tabular}
\end{center}
\end{table}

Drobnoustrój najczęściej spotykany razem z Escherichia coli 1 to Proteus mirabilis – ta kombinacja pojawiła się 66 razy.



\begin{table}[H]
  \begin{center}
  \caption{Wykaz badań dla materiału mocz dla najpopularniejszej kombinacji bakterii dla lat 1992-1996}
  \begin{tabular}{|c|c|c|c|}
    \hline Rok &Wspólnie & Kobiety & Mężczyźni \\ \hline
    Każdy& 66& 43& 23 \\ \hline 
    1992& 9& 5& 4 \\ \hline
    1993& 12& 7& 5 \\ \hline
    1994& 15& 10& 5 \\ \hline
    1995& 17& 13& 4 \\ \hline
    1996& 13& 8& 5 \\ \hline
    
  \end{tabular}
\end{center}
\end{table}

\begin{table}[H]
  \begin{center}
  \caption{Wykaz badań dla materiału mocz dla najpopularniejszej kombinacji bakterie dla oddziału nefrologii dla lat 1992-1996}
  \begin{tabular}{|c|c|c|c|}
    \hline Rok &Wspólnie & Kobiety & Mężczyźni \\ \hline
Każdy& 10& 7& 3 \\ \hline 
1992& 1& 0& 1 \\ \hline
1993& 1& 1& 0 \\ \hline
1994& 4& 2& 2 \\ \hline
1995& 2& 2& 0 \\ \hline
1996& 2& 2& 0 \\ \hline
  \end{tabular}
\end{center}
\end{table}

\end{document}