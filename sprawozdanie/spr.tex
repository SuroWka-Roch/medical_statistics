\documentclass[a4paper,12pt]{article}
\usepackage[utf8]{inputenc}
\usepackage[T1]{polski}
\usepackage{helvet}
\usepackage{graphicx}
\usepackage{color}
\usepackage{geometry}
\usepackage[T1]{fontenc}
\usepackage[unicode]{hyperref}
\usepackage{amsmath}
\usepackage{gensymb}
\usepackage{multirow}
\geometry{hmargin={2cm, 2cm}, height=10.0in}


\usepackage{lipsum} 
\usepackage{indentfirst}

\usepackage{listings}
\usepackage{color}

\definecolor{dkgreen}{rgb}{0,0.6,0}
\definecolor{gray}{rgb}{0.5,0.5,0.5}
\definecolor{mauve}{rgb}{0.58,0,0.82}

\lstset{frame=tb,
  language=Java,
  aboveskip=3mm,
  belowskip=3mm,
  showstringspaces=false,
  columns=flexible,
  basicstyle={\small\ttfamily},
  numbers=none,
  numberstyle=\tiny\color{gray},
  keywordstyle=\color{blue},
  commentstyle=\color{dkgreen},
  stringstyle=\color{mauve},
  breaklines=true,
  breakatwhitespace=true,
  tabsize=3
}

\title{Title}
\author{Wojciech Surówka}

\begin{document}
Dane główne:
\begin{table}[h]
\begin{center}
\begin{tabular}{cc}
\hline
Liczba pacjętów w bazie: & 6861 \\
liczba badń w bazie: & 17198 \\
liczba odnalezionych rodzajów drobnoustrojów & 15553 \\
Liczba wykonanych badań oporności & 136286  \\
\hline
\end{tabular}
\end{center}
\end{table}

Wykonano wstępną obróbkę danych:
\begin{itemize}
  \item Uwzględniono obecność pacjęta na wielu oddziałach,
  \item usunięto powtórzenia w zapisie pacjętów,
  \item usunięto badania o złym zapisie daty,
  \item uwzględniono badania w przedziale 1-1-1992 do 31-3-1996 włącznie.
  \item usunięto dane drobnoustrojstwa niezwiązanego z żadnym badaniem
  \item usunięto dane oporność niezwiązane z żadnym drobnoustrojstwem
\end{itemize}
Po przeprowadzeniu tych działań otrzymano zmienioną wielkość tablic:
\begin{table}[h]
\begin{center}
\begin{tabular}{c|c|c}
\hline
Nazwa danych & Nowa ilość & zmiana \\
Pacjęci & 6852 & 9 \\
Badania & 16995  & 203\\
Drobnoustrojstwa & 15421 & 132\\
Lekooporność & 135220 & 1066 \\
\hline
\end{tabular}
\end{center}
\end{table}

Na podstawie materiałów (pachwa, sperma) ustalono że w danych pacjęta 1 oznacza mężczyznę a 0 kobietę.

Ustalono że badano jedynie 41 rodzajów materiałów.

Wykryto 178 różnych drobnoustrojów.

Testowano 78 różnych leków.

W 8 polach pacjętów nie podano nazwy oddziału oraz nazwy wydziału są podawane wymiennie z małej i dużej litery.

\end{document}
