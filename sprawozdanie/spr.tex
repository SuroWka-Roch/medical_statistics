\documentclass[a4paper,12pt]{article}
\usepackage[utf8]{inputenc}
\usepackage[T1]{polski}
\usepackage{helvet}
\usepackage{graphicx}
\usepackage{color}
\usepackage{geometry}
\usepackage[T1]{fontenc}
\usepackage[unicode]{hyperref}
\usepackage{amsmath}
\usepackage{gensymb}
\usepackage{multirow}
\geometry{hmargin={2cm, 2cm}, height=10.0in}


\usepackage{lipsum} 
\usepackage{indentfirst}

\usepackage{listings}
\usepackage{color}

\definecolor{dkgreen}{rgb}{0,0.6,0}
\definecolor{gray}{rgb}{0.5,0.5,0.5}
\definecolor{mauve}{rgb}{0.58,0,0.82}

\lstset{frame=tb,
  language=Java,
  aboveskip=3mm,
  belowskip=3mm,
  showstringspaces=false,
  columns=flexible,
  basicstyle={\small\ttfamily},
  numbers=none,
  numberstyle=\tiny\color{gray},
  keywordstyle=\color{blue},
  commentstyle=\color{dkgreen},
  stringstyle=\color{mauve},
  breaklines=true,
  breakatwhitespace=true,
  tabsize=3
}

\title{Title}
\author{Wojciech Surówka}

\begin{document}

\begin{table}[t]
\centering
\label{my-label}
\resizebox{\textwidth}{!}{
\begin{tabular}{|c|c|l|l|l|l|}
\hline

\begin{tabular}[c]{@{}c@{}}Wydział: FiIS\\ Kierunek: FM\end{tabular} & \begin{tabular}[c]{@{}c@{}}Rok: II\\ Zespół: ?\end{tabular} & \multicolumn{4}{l|}{\begin{tabular}[c]{@{}l@{}}Filip Kosiorowki \\ Wojciech Surówka\end{tabular}} \\ \hline

\textbf{\begin{tabular}[c]{@{}c@{}}Statystyka medyczna\end{tabular}} & \multicolumn{5}{l|}{\begin{tabular}[c]{@{}l@{}}Temat:Promieniowanie niejonizujące sieci komórkowej\end{tabular}} \\ \hline

\begin{tabular}[c]{@{}c@{}}Data wykonania:\\ 1.11.2019 r.\end{tabular} & \begin{tabular}[c]{@{}c@{}}Data oddania:\\ 1.11.2019 r.\end{tabular} & \begin{tabular}[c]{@{}l@{}}Zwrot do poprawy:\\\end{tabular} & \begin{tabular}[ct]{@{}l@{}}Data oddania poprawy:\\\end{tabular} & \begin{tabular}[cT]{@{}l@{}}Data zaliczenia:\\\end{tabular} & \begin{tabular}[cT]{@{}l@{}}Ocena:\end{tabular} \\ \hline

\end{tabular}}
\end{table}

Dane główne:
\begin{table}[h]
\begin{center}
\caption{Główne informacje o bazie danych}
\begin{tabular}{cc}
\hline
Liczba pacjętów w bazie: & 6861 \\
liczba badń w bazie: & 17198 \\
liczba odnalezionych rodzajów & 15553 \\
Liczba wykonanych badań oporności & 136286  \\
\hline
\end{tabular}
\end{center}
\end{table}

Wykonano wstępną obróbkę danych:
\begin{itemize}
  \item Uwzględniono obecność pacjęta na wielu oddziałach (7),
  \item usunięto powtórzenia w zapisie pacjętów (2), 
  \item usunięto badania o złym zapisie daty,
  \item uwzględniono badania w przedziale 1-1-1992 do 31-3-1996 włącznie.
  \item usunięto dane drobnoustrojstwa niezwiązanego z żadnym badaniem
  \item usunięto dane oporność niezwiązane z żadnym drobnoustrojstwem
\end{itemize}

Po przeprowadzeniu tych działań otrzymano zmienioną wielkość tablic:
\begin{table}[h]
\begin{center}
\caption{Zmiany wielkości danych po wstępnej filtracji}
\begin{tabular}{c|c|c}
\hline
Nazwa danych & Nowa ilość & zmiana \\
Pacjęci & 6852 & 9 \\
Badania & 16995  & 203\\
Drobnoustrojstwa & 15421 & 132\\
Lekooporność & 135220 & 1066 \\
\hline
\end{tabular}
\end{center}
\end{table}

Na podstawie materiałów (pachwa, sperma) ustalono że w danych pacjęta 1 oznacza mężczyznę a 0 kobietę.

Ustalono że badano jedynie 41 rodzajów materiałów.

Wykryto 178 różnych drobnoustrojów.

Testowano 78 różnych leków.

W 8 polach pacjętów nie podano nazwy oddziału oraz nazwy wydziału są podawane wymiennie z małej i dużej litery.
W celu unifikacji wszelkie wydziały z małej litery zapisano w samych dużych literach.

Wyniki z wstempnego szacowania umieszczono w tabelach poniżej.

\begin{table}[h]
\begin{center}
\begin{tabular}{|c|c|c|}
\hline
Typ pacjętów &  co najmniej jedno badanie  & co najmniej jeden drobnoustrój \\
\hline
Wszyscy & 6746 &5189\\
kobiety  & 3165  & 2428\\
Mężczyżni & 3581 & 2761\\ \hline
\multicolumn{3}{|c|}{Rok 1992} \\ \hline
Wszyscy & 2150 &1599\\
kobiety  & 958  & 698\\
Mężczyżni & 1192 & 901\\ \hline
\multicolumn{3}{|c|}{Rok 1993} \\ \hline
Wszyscy & 1367 &1282\\
kobiety  & 690  & 630\\
Mężczyżni & 677 & 652\\ \hline
\multicolumn{3}{|c|}{Rok 1994} \\ \hline
Wszyscy & 1350 &1173\\
kobiety  & 646  & 563\\
Mężczyżni & 704 & 610\\ \hline
\multicolumn{3}{|c|}{Rok 1995} \\ \hline
Wszyscy & 1864 &1254\\
kobiety  & 888  & 613\\
Mężczyżni & 976 & 641\\ \hline
\multicolumn{3}{|c|}{Rok 1996} \\ \hline
Wszyscy & 559 &405\\
kobiety  & 258  & 191\\
Mężczyżni & 301 & 214\\ \hline
\hline
\end{tabular}
\end{center}
\end{table}

\begin{table}[h]
\begin{center}
\caption{"Wykaz badań dla wszystkich oddziałów"}
\begin{tabular}{c|c|c}
\hline
Rok badań & Ilość badań & Ilość badań \\
& z przynajmniej jednym drobnoustrojem &z testem lekooporności \\
1992&5106&2327 \\
1993&3097&1992 \\
1994&3186&2120 \\
1995&4346&2126 \\
1996&1260&704 \\
\hline
\end{tabular}
\end{center}
\end{table}

\begin{table}[h]
\begin{center}
\caption{"Wykaz badań dla poszczególnych oddziałów"}
\begin{tabular}{c|c|c}
\hline
Rok badań & Ilość badań & Ilość badań \\
& z przynajmniej jednym drobnoustrojem &z testem lekooporności \\
\hline \multicolumn{3}{|c|}{Oddział OIOM} \\ \hline
wspólnie &6043 &3546 \\
1992 &1649 &893 \\
1993 &1210 &785 \\
1994 &1297 &843 \\
1995 &1447 &778 \\
1996 &440 &247 \\
\hline \multicolumn{3}{|c|}{Oddział NACZ} \\ \hline
wspólnie &428 &304 \\
1992 &160 &99 \\
1993 &123 &97 \\
1994 &51 &50 \\
1995 &79 &49 \\
1996 &15 &9 \\
\hline \multicolumn{3}{|c|}{Oddział URAZ} \\ \hline
wspólnie &1171 &683 \\
1992 &308 &139 \\
1993 &191 &144 \\
1994 &231 &174 \\
1995 &356 &176 \\
1996 &85 &50 \\
\hline \multicolumn{3}{|c|}{Oddział WEWN} \\ \hline
wspólnie &1848 &623 \\
1992 &822 &232 \\
1993 &251 &108 \\
1994 &293 &118 \\
1995 &401 &127 \\
1996 &81 &38 \\
\hline \multicolumn{3}{|c|}{Oddział CHIR} \\ \hline
wspólnie &622 &422 \\
1992 &180 &107 \\
1993 &87 &74 \\
1994 &125 &100 \\
1995 &179 &100 \\
1996 &51 &41 \\
\hline \multicolumn{3}{|c|}{Oddział REUM} \\ \hline
wspólnie &799 &372 \\
1992 &204 &73 \\
1993 &160 &88 \\
1994 &171 &107 \\
1995 &200 &78 \\
1996 &64 &26 \\

\hline
\end{tabular}
\end{center}
\end{table}

\begin{table}[h]
\begin{center}
\caption{"Wykaz badań dla poszczególnych oddziałów c.d."}
\begin{tabular}{c|c|c}
\hline
Rok badań & Ilość badań & Ilość badań \\
& z przynajmniej jednym drobnoustrojem &z testem lekooporności \\

\hline \multicolumn{3}{|c|}{Oddział NEFR} \\ \hline
wspólnie &2039 &789 \\
1992 &511 &125 \\
1993 &343 &151 \\
1994 &312 &193 \\
1995 &693 &241 \\
1996 &180 &79 \\
\hline \multicolumn{3}{|c|}{Oddział UROL} \\ \hline
wspólnie &552 &320 \\
1992 &178 &83 \\
1993 &123 &87 \\
1994 &85 &59 \\
1995 &130 &65 \\
1996 &36 &26 \\
\hline \multicolumn{3}{|c|}{Oddział NECH} \\ \hline
wspólnie &649 &390 \\
1992 &195 &98 \\
1993 &117 &97 \\
1994 &122 &89 \\
1995 &148 &75 \\
1996 &67 &31 \\
\hline \multicolumn{3}{|c|}{Oddział NEUR} \\ \hline
wspólnie &681 &392 \\
1992 &290 &133 \\
1993 &161 &119 \\
1994 &85 &68 \\
1995 &105 &53 \\
1996 &40 &19 \\


\end{tabular}
\end{center}
\end{table}

\begin{table}[h]
\begin{center}
\caption{"Wykaz badań dla poszczególnych oddziałów c.d. "}
\begin{tabular}{c|c|c}
\hline
Rok badań & Ilość badań & Ilość badań \\
& z przynajmniej jednym drobnoustrojem &z testem lekooporności \\

\hline \multicolumn{3}{|c|}{Oddział USPR} \\ \hline
wspólnie &846 &555 \\
1992 &180 &103 \\
1993 &95 &70 \\
1994 &200 &139 \\
1995 &264 &174 \\
1996 &107 &69 \\
\hline \multicolumn{3}{|c|}{Oddział PLAS} \\ \hline
wspólnie &381 &286 \\
1992 &141 &95 \\
1993 &81 &66 \\
1994 &57 &51 \\
1995 &83 &59 \\
1996 &19 &15 \\
\hline \multicolumn{3}{|c|}{Oddział OKUL} \\ \hline
wspólnie &204 &101 \\
1992 &50 &17 \\
1993 &23 &10 \\
1994 &20 &14 \\
1995 &81 &40 \\
1996 &30 &20 \\
\hline \multicolumn{3}{|c|}{Oddział LARY} \\ \hline
wspólnie &605 &407 \\
1992 &214 &122 \\
1993 &110 &80 \\
1994 &103 &93 \\
1995 &140 &84 \\
1996 &38 &28 \\
\hline \multicolumn{3}{|c|}{Oddział SZCZ} \\ \hline
wspólnie &153 &95 \\
1992 &36 &17 \\
1993 &31 &20 \\
1994 &38 &25 \\
1995 &41 &27 \\
1996 &7 &6 \\

\end{tabular}
\end{center}
\end{table}

\end{document}
